\documentclass[../../master.tex]{subfiles}

\begin{document}
\section{Fundamental Concepts}

% Exercise 1.1
\begin{exercise}
    Check the distributive laws for $\cup$ and $\cap$ and DeMorgan's laws.
\end{exercise}

\begin{solution}
    First, we prove that $A \cap (B \cup C) = (A \cap B) \cup (A \cap C)$. 
    Let $x \in A \cap (B \cup C)$. 
    Then $x \in A$ and $x \in B$ or $x \in C$.
    But this implies that $x \in A \cap B$ or $x \in A \cap C$. 
    Therefore, we have $x \in (A \cap B) \cup (A \cap C)$. 
    Thus, $A \cap (B \cup C) \subseteq (A \cap B) \cup (A \cap C)$.

    For the opposite direction, let $x \in (A \cap B) \cup (A \cap C)$. 
    This implies that $x \in A \cap B$ or $x \in A \cap C$. 
    In either case, $x \in A$. 
    Furthermore, we have $x \in B$ or $x \in C$. 
    But then $x \in A \cap (B \cup C)$ so $(A \cap B) \cup (A \cap C) \subseteq A \cap (B \cup C)$, proving the equality.
    The proof for the second statement is entirely analogous.

    Now we will prove the first of DeMorgan's laws, that $A - (B \cup C) = (A - B) \cap (A - C)$.
    Let $x \in A - (B \cup C)$. 
    That is, $x \in A$ and $x \notin B \cup C$. 
    But then $x \notin B$ and $x \notin C$ so $x \in A - B$ and $x \in A - C$.
    Thus, $x \in (A - B) \cap (A - C)$ so $A - (B \cup C) \subseteq (A - B) \cap (A - C)$.

    For the other direction, let $x \in (A - B) \cap (A - C)$. 
    Then $x \in A - B$ and $x \in A - C$ so $x \in A$, $x \notin B$, and $x \notin C$.
    But then $x \notin B \cup C$ so $x \in A - (B \cup C)$.
    Therefore, $(A - B) \cap (A - C) \subseteq A - (B \cup C)$, proving equality of sets.
    The proof of the second statement is again analogous.
\end{solution}

% Exercise 1.2
\begin{exercise}
    Determine which of the following statements are true for all sets $A, B, C$, and $D$.
    If a double implication fails, determine whether one or the other of the possible implications holds.
    If an equality fails, determine whether the statement becomes true if the "equals" symbol is replaced by one or the other of the inclusion symbols $\subset$ or $\supset$.
    \begin{enumerate}[label=(\alph*)]
        \item $A \subset B$ and $A \subset C \Leftrightarrow A \subset (B \cup C)$.
        \item $A \subset B$ or $A \subset C \Leftrightarrow A \subset (B \cup C)$.
        \item $A \subset B$ and $A \subset C \Leftrightarrow A \subset (B \cap C)$.
        \item $A \subset B$ or $A \subset C \Leftrightarrow A \subset (B \cap C)$.
        \item $A - (A - B) = B$.
        \item $A - (B - A) = A - B$.
        \item $A \cap (B - C) = (A \cap B) - (A \cap C)$.
    \end{enumerate}
\end{exercise}

\begin{solution}
    The forward direction of (a) is true but the reverse direction is false.
    To see this, let $a \in A$.
    Then $a \in B$ and $a \in C$ so $a \in B \cup C$.
    Therefore, $A \subset (B \cup C)$.
    However, consider $A \subset B$ and $A \not\subset C$.
    Certainly $A \subset B \cup C$ but the left side does not hold.

    The forward direction of (b) is true but the reverse direction is false.
    To prove the forward direction, let $a \in A$.
    If $A \subset B$, then $a \in B$ so $a \in B \cup C$.
    If $A \subset C$, then $a \in B$ and still we have $a \in B \cup C$.
    In either case, $A \subset (B \cup C)$.
    For a counterexample to the reverse direction, consider $A = \{1, 2\}$, $B = \{1\}$, $C = \{2, 3\}$.
    Clearly $A \subset B \cup C$ but $A \not\subset B$ and $A \not\subset C$.

    Both directions of (c) are true.
    Indeed, suppose $A \subset B$ and $A \subset C$ and let $a \in A$.
    Then $a \in B$ and $a \in C$ so $a \in B \cap C$, showing that $A \subset (B \cap C)$.
    Now suppose $A \subset (B \cap C)$ and let $a \in A$.
    Then $a \in B \cap C$ so $a \in B$ and $a \in C$. 
    This implies that $A \subset B$ and $A \subset C$.

    The forward direction of (d) is false but the reverse direction is true.
    To prove the reverse direction, suppose $A \subset (B \cap C)$ and let $a \in A$.
    Then $a \in B \cap C$ so $a \in B$ and $a \in C$, satisfying the left side of the statement.
    For a counterexample to the forward direction, consider $A = \{0, 1\}$, $B = \{0, 1, 2\}$, and $C = \{2\}$. 
    Then $A \subset B$ so the left side is satisfied by $B \cap C = \{2\}$ which $A$ is not a subset of.

    Statement (e) holds if the equality is replaced with $\subset$.
    To see this, let $x \in A - (A - B)$. 
    Then $x \in A$ but $x \notin A - B$. 
    That is, $x \in B$ so $A - (A - B) \subset B$.
    For a counterexample to the other direction, consider the sets $A = \{0, 1\}$ and $B = \{1, 2\}$.
    Then $A - (A - B) = \{1\} \neq B$, showing equality does not hold.

    Statement (f) holds if equality is replaced with $\supset$.
    Let $x \in A - B$. 
    Then $x \in A$ and $x \notin B$. 
    But if $x \notin B$ then $x \notin B - A$.
    Thus, $x \in A - (B - A)$ so $A - (B - A) \supset A - B$.
    For a counterexample to the other direction, consider $A = \{0, 1\}$ and $B = \{1, 2\}$. 
    Then $A - (B - A) = \{0, 1\}$ while $A - B = \{0\}$, showing equality does not hold.

    The equality in statement (g) holds.
    To prove this, let $x \in A \cap (B - C)$.
    Then $x \in A$ and $x \in B$, but $x \notin C$.
    That is, $x \in A \cap B$ and $x \notin A \cap C$.
    Thus, $x \in (A \cap B) - (A \cap C)$, showing that $\subset$ holds.
    Now let $x \in (A \cap B) - (A \cap C)$.
    This implies that $x \in A \cap B)$ but $x \notin A \cap C$.
    The first part shows that $x \in A$ and $x \in B$, so the second statement implies $x \notin C$.
    But then $x \in B - C$ so $x \in A \cap (B - C)$ and $\supset$ holds.
    Since both sides are subsets of each other, equality holds.
\end{solution}
\end{document}
